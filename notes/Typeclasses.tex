\chapter{Typeclasses}
%================================================
\section{Introduction}
Typeclasses offer a mechanism for dealing with {\em ad-hoc} polymorphism which ``occurs when a function is defined over several different types, acting in a different way for each type''  in contrast to  {\em parametric} polymorphism, which happens when a function is defined over a range of types and acts in  the {\em same} way  for each type i.e. the \texttt{length} function.  \citep {WADLER-BLOTT}

Typeclasses may be ``thought of as a kind of bounded quantifier, limiting the types that a type variable may instatiate to'' where {\em type coercion} is not allowed. They may also be thought of as {\em abstract data types} where each type specifies a number of methods (functions) but does not say how they are to be implemented (that is left up to the individual types).  \citep {WADLER-BLOTT}

%==========================================================
\section{Declaring Typeclasses}
Typeclass names and constructors begin with a capital letter, i.e.  \texttt{Num, Real, Float} while {\em type variables} are given by lower case letters i.e. \texttt{a, b, ..}.

A typeclass is declared to have one or more methods (functions); the methods may, or may not, have default implementations. For example, if the \texttt{Num} typeclass has the following declaration:
\begin{lstlisting}
	class Num a where
		(+), (*) :: a -> a -> a
		negate   :: a -> a
\end{lstlisting}
it essentially declares that any type \texttt{a} belonging to the typeclass \texttt{Num} will have an {\em instance} declaration that provides implementations for each of the typeclass \texttt{Num}'s declared methods. For example, the \texttt{Int} type is added to the \texttt{Num} typeclass by declaring an \texttt{instance} as follows:
\begin{lstlisting}
	instance Num Int where
		(+)    = addInt    
		(*)    = mulInt
		negate = negateInt
\end{lstlisting}
where the type \texttt{Int} replaces the type variable, \texttt{a}, and \texttt{addInt, mulInt} and \texttt{negateInt} are all functions that perform the required behaviour.

The standard prelude includes similar instances for all the number types so that \texttt{+,*,/}, etc. work across all \texttt{Num} types.

To see the full declaration for any typeclass, load up \texttt{ghci} and enter \texttt{:info} or \texttt{:i}
\begin{lstlisting}
	Prelude> :info Num
    class Num a where
      (+)         :: a -> a -> a
      (*)         :: a -> a -> a
      (-)         :: a -> a -> a
      negate      :: a -> a
      abs         :: a -> a
      signum      :: a -> a
      fromInteger :: Integer -> a
        -- Defined in 'GHC.Num'
    instance Num Integer -- Defined in 'GHC.Num'
    instance Num Int     -- Defined in 'GHC.Num'
    instance Num Float   -- Defined in 'GHC.Float'
    instance Num Double  -- Defined in 'GHC.Float'
\end{lstlisting}
%===========================================================
\section {Adding Class Constraints (Subclasses)}
If we want every member of a typeclass to also be a member of another typeclass we can add a {\em typeclass constraint} to the typeclass declaration. For example, the \texttt{Integral} class is declared as:
\begin{lstlisting}
	class (Real a, Enum a) => Integral a where
		...
\end{lstlisting}
which states that any type \texttt{a} that belongs to the \texttt{Integral} typeclass must also belong to the \texttt{Real} and \texttt{Enum} typeclasses. And that essentially means that {\em all} \texttt{Integral} types also have the behaviours (methods) of all \texttt{Real} and \texttt{Enum} types.
